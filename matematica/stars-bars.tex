É uma técnica matemática para resolver certos problemas combinatórios. Ocorre sempre que você deseja contar o número de maneiras de agrupar objetos idênticos.
 
xx $\vert$ xxx $\vert$ xxx, tal que x são as estrelas e $\vert$ são as barras

\begin{itemize}
  \item Ferramenta de combinatória que é utilizada quando os itens são indistinguíveis
  \item O número de maneiras de colocar n objetos idênticos em k boxes é: nCk(n+k-1, k-1) ou nCk(n+k-1, n)
  \item Você pode imaginar n+k-1 slots finais, n estrelas + k-1 barras (já que teremos k grupos), onde você quer preencher k-1 (ou n) desses slots com barras e o resto dos slots que sobrarem com estrelas.
  \item Para considerar de forma que cada box contenha pelo menos um idem: nCk(n-1, k-1)
\end{itemize}

%----DAQUI EM DIANTE FOI TUDO RETIRADO DO CP ALGORITHMS----
\subsubsection*{Stars and bars (texto daqui em diante retirado do CP Algorithms)}

Stars and bars is a mathematical technique for solving certain combinatorial problems.
It occurs whenever you want to count the number of ways to group identical objects.

\subsubsection*{Theorem}

The number of ways to put $n$ identical objects into $k$ labeled boxes is

$$\binom{n + k - 1}{n}.$$

The proof involves turning the objects into stars and separating the boxes using bars (therefore the name).
E.g. we can represent with $\bigstar | \bigstar \bigstar |~| \bigstar \bigstar$ the following situation:
in the first box is one object, in the second box are two objects, the third one is empty and in the last box are two objects.
This is one way of dividing 5 objects into 4 boxes.

It should be pretty obvious, that every partition can be represented using $n$ stars and $k - 1$ bars and every stars and bars permutation using $n$ stars and $k - 1$ bars represents one partition.
Therefore the number of ways to divide $n$ identical objects into $k$ labeled boxes is the same number as there are permutations of $n$ stars and $k - 1$ bars.
The Binomial Coefficient gives us the desired formula.

\subsubsection*{Number of non-negative integer sums}

This problem is a direct application of the theorem.

You want to count the number of solution of the equation 

$$x_1 + x_2 + \dots + x_k = n$$

with $x_i \ge 0$.

Again we can represent a solution using stars and bars.
E.g. the solution $1 + 3 + 0 = 4$ for $n = 4$, $k = 3$ can be represented using $\bigstar | \bigstar \bigstar \bigstar |$.

It is easy to see, that this is exactly the stars and bars theorem.
Therefore the solution is $\binom{n + k - 1}{n}$.

\subsubsection*{Number of positive integer sums}

A second theorem provides a nice interpretation for positive integers. Consider solutions to 

$$x_1 + x_2 + \dots + x_k = n$$

with $x_i \ge 1$.

We can consider $n$ stars, but this time we can put at most \textit{one bar} between stars, since two bars between stars would represent $x_i=0$, i.e. an empty box. 
There are $n-1$ gaps between stars to place $k-1$ bars, so the solution is $\binom{n-1}{k-1}$. 

\subsubsection*{Number of lower-bound integer sums}

This can easily be extended to integer sums with different lower bounds.
I.e. we want to count the number of solutions for  the equation

$$x_1 + x_2 + \dots + x_k = n$$

with $x_i \ge a_i$.

After substituting $x_i' := x_i - a_i$ we receive the modified equation

$$(x_1' + a_i) + (x_2' + a_i) + \dots + (x_k' + a_k) = n$$

$$\Leftrightarrow ~ ~ x_1' + x_2' + \dots + x_k' = n - a_1 - a_2 - \dots - a_k$$

with $x_i' \ge 0$.
So we have reduced the problem to the simpler case with $x_i' \ge 0$ and again can apply the stars and bars theorem.

\subsubsection*{Number of upper-bound integer sums}

With some help of the Inclusion-Exclusion Principle, you can also restrict the integers with upper bounds.
See the Number of upper-bound integer sums section in the corresponding article.
%-------------------------------